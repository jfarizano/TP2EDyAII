\documentclass[11pt]{article}
\usepackage[a4paper, margin=2.54cm]{geometry}
\usepackage[utf8]{inputenc}
\usepackage[spanish, mexico]{babel}
\usepackage[spanish]{layout}
\usepackage[article]{ragged2e}
\usepackage{textcomp}
\usepackage{amsmath}
\usepackage{amsfonts}
\setlength{\parindent}{0pt}


\title{Trabajo Práctico 2: Análisis de Costos}

\author{Farizano, Juan Ignacio \and Mellino, Natalia}
\date{}

\begin{document}
\maketitle

\section*{Implementación de Secuencias con Listas}

%==============================================================================

\subsection*{Función mapS}

(completar xd)

%==============================================================================

\subsection*{Función appendS}

\textbf{\underline{Trabajo:}} \\

Sean $n$ y $m$ las longitudes de las listas que recibe como argumento
la función \texttt{appendS}, podemos ver que la recurrencia para el
trabajo nos queda expresada como:

\begin{equation*}
    W(n + m) = W(n + m - 1) + k
\end{equation*}

Donde $k$ es una constante. \\

Podemos demostrar fácilmente por inducción que $W \in O(n+m)$:

\begin{align*}
    W(n + m) & = W(n + m - 1) + k \\
             & \leq c(n + m - 1) + k \rightarrow \text{HI}\\
             & \leq c(n + m) - c + k \\
             & \leq c(n + m) \iff c \geq k
\end{align*}

Por lo tanto $W \in O(n + m)$ \\

\textbf{\underline{Profundidad:}} \\

Para la profundidad, la recurrencia nos queda expresada igual que la del
trabajo:

\begin{equation*}
    S(n + m) = S(n + m - 1) + k \; \; \in O(n + m)
\end{equation*}

Es decir, tanto el trabajo como la profundidad de la función \texttt{appendS}
son del orden de la suma de la longitud de ambas listas.

%==============================================================================

\subsection*{Función reduceS}

\textbf{\underline{Trabajo:}} \\

La recurrencia para \texttt{reduceS} la podemos expresar de la siguiente manera
(recordemos que se asume que la función que recibe como argumento es de orden constante):
\begin{equation*}
    W(n) = W(\frac{n}{2}) + W_{contract}(n) + k
\end{equation*}

Ahora necesitamos saber que orden tiene $W_{contract}(n)$, observemos que su 
recurrencia es de la forma:

\begin{equation*}
    W_{contract(n)} = W_{contract}(n-2) + k 
\end{equation*}

Podemos demostrar que $W_{contract} \in O(n)$:

\begin{align*}
    W_{contract}(n) & = W_{contract}(n - 2) + k \\
             & \leq c(n - 2) + k \rightarrow \text{HI}\\
             & \leq cn - 2c + k \\
             & \leq cn \iff c \geq \frac{k}{2}
\end{align*}

Ahora, utilizando el tercer caso del \textbf{Teorema Maestro} podemos probar que
$W(n) \in O(n \;lg\;n)$, debemos ver dos cosas: \\

Sean $a = 1, \; b = 2$

\begin{itemize}
    \item Existe $\epsilon > 0$ tal que $f(n) \in \Omega(n^{lg_2 1 + \epsilon})$:
          de hecho, como $f(n)$ es $O(n)$ basta tomar $\epsilon = 1$
          y trivialmente se satisface la condición
    \item  Existe $c < 1$ y $N \in \mathbb{N}$ tal que para todo $n > N$,
           $1 \cdot f(\frac{n}{2}) \leq c \cdot f(n)$: nuevamente, como $f(n)$ es $O(n)$, 
           podemos tomar $c = \frac{1}{2}$ y $N = 1$ y se cumple: $\frac{n}{2} \leq \frac{1}{2}n$. 
\end{itemize}

Entonces, como se cumplen las hipótesis del Teorema, podemos decir que $W \in O(f(n))$ y
como $f(n) \in O(n)$, por transitividad, resulta $W \in O(n)$. \\

\textbf{\underline{Profundidad:}} \\
xd
%==============================================================================

\subsection*{Función scanS}

(completar xd)

%==============================================================================
%==============================================================================

\section*{Implementación de Secuencias con Arreglos}

\subsection*{Función mapS}
(completar xd)

\subsection*{Función appendS}
(completar xd)

\subsection*{Función reduceS}
(completar xd)

\subsection*{Función scanS}
(completar xd)

\end{document}