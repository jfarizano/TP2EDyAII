\documentclass[11pt]{article}
\usepackage[a4paper, margin=2.54cm]{geometry}
\usepackage[utf8]{inputenc}
\usepackage[spanish, mexico]{babel}
\usepackage[spanish]{layout}
\usepackage[article]{ragged2e}
\usepackage{textcomp}
\usepackage{amsmath}
\usepackage{amsfonts}
\setlength{\parindent}{0pt}


\title{Trabajo Práctico 2 EDyAII \\
        \large Análisis de Costos
}

\author{Farizano, Juan Ignacio \and Mellino, Natalia}
\date{}

%==============================================================================
%==============================================================================
%==============================================================================

\begin{document}
\maketitle

\noindent\rule{\textwidth}{1pt}
\section{Implementación de Secuencias con Listas}

%==============================================================================

\subsection{Función mapS}
Sea $xs = [x_{\vert xs \vert -1},...,x_0]$ una lista, $n = \vert xs \vert$ su longitud
y sea $f$ la función que toma \texttt{mapS} como argumento. 

\subsubsection{Trabajo:}
La recurrencia para el trabajo la podemos expresar de la siguiente manera:

\begin{equation*}
    W_{mapS}(n) = W_f(x_{n- 1}) + W_{mapS}(n - 1) + k
\end{equation*}

Donde $k$ es una constante, y $x$ es la cabeza de la lista. \\

Demostramos por inducción que $ W_{mapS} \in O\left(\displaystyle\sum_{i=0}^{n - 1}W_f(x_i) - k\right) $

\begin{align*}
    W_{mapS}(n) & = W_f(x_{n - 1}) + W_{mapS}(n - 1) + k \\
               & \leq W_f(x_{n - 1}) + c \cdot (\displaystyle\sum_{i=0}^{n - 2}W_f(x_i) - k) + k \rightarrow \text{HI} \\
               & \leq c \cdot  W_f(x_{n - 1}) + c \cdot \displaystyle\sum_{i=0}^{n - 2}W_f(x_i) - c \cdot k + k \\
               & \leq c \cdot \displaystyle\sum_{i=0}^{n - 1}W_f(x_i) \iff c \geq 1
\end{align*}

Por lo tanto como $ W_{mapS} \in O\left(\displaystyle\sum_{i=0}^{n - 1}W_f(x_i) - k\right) $,
luego podemos decir que $ W_{mapS} \in O\left(\displaystyle\sum_{i=0}^{n - 1}W_f(x_i)\right) $

\subsubsection{Profundidad:}

Para la profundidad, la recurrencia la podemos expresar de la siguiente manera:

\begin{equation*}
    S_{mapS}(n) = max(S_f(x_{n- 1}), S_{mapS}(n - 1)) + k
\end{equation*}

Podemos demostrar también que $ S_{mapS} \in O\left(\displaystyle\sum_{i=0}^{n - 1}S_f(x_i)\right) $:

\begin{align*}
    S_{mapS}(n) & = max(S_f(x_{n- 1}), S_{mapS}(n - 1)) + k \\
               & \leq S_f(x_{n- 1}) + S_{mapS}(n - 1) + k
\end{align*}

Si observamos la ecuación anterior, podemos ver que se obtuvo una similar
a la del trabajo, por lo tanto, el análisis de la profundidad se realiza
de la misma manera que en el trabajo. Entonces podemos concluir que 
$ S_{mapS} \in O\left(\displaystyle\sum_{i=0}^{n - 1}S_f(x_i)\right) $.

%==============================================================================

\subsection{Función appendS}

Sean $n$ la longitud de la primer lista que recibe como argumento la función \texttt{appendS}.
Observemos, que tanto el trabajo como la profundidad se realizan con respecto $n$
ya que es la lista que vamos consumiendo en cada llamada recursiva.
 
\subsubsection{Trabajo:}
Podemos ver que la recurrencia para el trabajo nos queda expresada como:

\begin{equation*}
    W_{appendS}(n) = W_{appendS}(n - 1) + k
\end{equation*}

Donde $k$ es una constante. \\

Podemos demostrar fácilmente por inducción que $W \in O(n)$:

\begin{align*}
    W_{appendS}(n) & = W_{appendS}(n - 1) + k \\
             & \leq c(n - 1) + k \rightarrow \text{HI}\\
             & \leq c \cdot n - c + k \\
             & \leq c \cdot n \iff c \geq k
\end{align*}

Por lo tanto $W_{appendS} \in O(n)$ \\

\subsubsection{Profundidad:}

Para la profundidad, la recurrencia nos queda expresada igual que la del
trabajo:

\begin{equation*}
    S_{appendS}(n) = S_{appendS}(n - 1) + k \; \; \in O(n)
\end{equation*}

Es decir, tanto el trabajo como la profundidad de la función \texttt{appendS}
son del orden de la suma de la longitud de ambas listas.

%==============================================================================

\subsection{Función reduceS}
Sea $n$ la longitud de la lista que \texttt{reduceS} recibe como argumento.
\subsubsection{Trabajo:}

La recurrencia para \texttt{reduceS} la podemos expresar de la siguiente manera
(recordemos que se asume que la función que recibe como argumento es de orden constante):
\begin{equation*}
    W_{reduceS}(n) = W(\frac{n}{2}) + W_{contract}(n) + k
\end{equation*}

Ahora necesitamos saber que orden tiene $W_{contract}(n)$, observemos que su 
recurrencia es de la forma:

\begin{equation*}
    W_{contract(n)} = W_{contract}(n-2) + k 
\end{equation*}

Podemos demostrar que $W_{contract} \in O(n)$:

\begin{align*}
    W_{contract}(n) & = W_{contract}(n - 2) + k \\
             & \leq c(n - 2) + k \rightarrow \text{HI}\\
             & \leq cn - 2c + k \\
             & \leq cn \iff c \geq \frac{k}{2}
\end{align*}

Ahora, utilizando el tercer caso del \textbf{Teorema Maestro} podemos probar que
$W_{reduceS}(n) \in O(n)$, debemos ver dos cosas: \\

Sean $a = 1, \; b = 2$ y $f(n) = W_{contract}(n)$

\begin{itemize}
    \item Existe $\epsilon > 0$ tal que $f(n) \in \Omega(n^{lg_2 1 + \epsilon})$:
          de hecho, como $f(n)$ es $O(n)$ basta tomar $\epsilon = 1$
          y trivialmente se satisface la condición
    \item  Existe $c < 1$ y $N \in \mathbb{N}$ tal que para todo $n > N$,
           $1 \cdot f(\frac{n}{2}) \leq c \cdot f(n)$: nuevamente, como $f(n)$ es $O(n)$, 
           podemos tomar $c = \frac{1}{2}$ y $N = 1$ y se cumple: $\frac{n}{2} \leq \frac{1}{2}n$. 
\end{itemize}

Entonces, como se cumplen las hipótesis del caso mencionado,
podemos decir que $W_{reduceS} \in O(f(n))$ y como $f(n) \in O(n)$, por transitividad,
resulta $W_{reduceS} \in O(n)$. \\

\subsubsection{Profundidad:}

Para la profundidad tenemos la siguiente recurrencia:

\begin{equation*}
    S_{reduceS}(n) = max(S_{contract}(n), S_{reduceS}(\frac{n}{2})) + k
\end{equation*}

Podemos ver que también, $S \in O(n)$:

\begin{align*}
    S_{reduceS}(n) & = max(S_{contract}(n), S_{reduceS}(\frac{n}{2})) + k \\
                   & \leq S_{contract}(n) + S_{reduceS}(\frac{n}{2}) + k
\end{align*}

Observemos que ahora la recurrencia nos quedo expresada de manera similar
a la del trabajo, por lo tanto, viendo el análisis anterior podemos 
concluir que $S_{reduceS} \in O(n)$.

%==============================================================================

\subsection{Función scanS}
\subsubsection{Trabajo:}

\subsubsection{Profundidad:}

%==============================================================================
%==============================================================================
%==============================================================================

\section{Implementación de Secuencias con Arreglos}

Por especificación tenemos que
\begin{align*}
& W_{tabulate}(f \; n) \in O\left(\displaystyle\sum_{i=0}^{n - 1}W_f(i)\right) \\
& S_{tabulate}(f \; n) \in O\left(\displaystyle\max_{i=0}^{n - 1}S_f(i)\right)
\end{align*}

\subsection{Función mapS}

Sea $f$ la función que se recibe como argumento, y $n$ la longitud del arreglo
sobre el cual se evaluará $f$ sobre sus elementos.
\subsubsection{Trabajo:}
 
La recurrencia para el trabajo de la función \texttt{mapS} la podemos
expresar como sigue:

\begin{equation*}
    W_{mapS}(f \; n) = W_{tabulate}(f \; n) + \underbrace{W_{length}(n)}_{\in \; O(1)} + k
\end{equation*} 

Por lo tanto $ W_{mapS}(f \; n) \in O\left(\displaystyle\sum_{i=0}^{n - 1}W_f(i)\right) $

\subsubsection{Profundidad:}

Luego, para la profundidad tenemos la siguiente recurrencia:

\begin{equation*}
    S_{mapS}(f \; n) = S_{tabulate}(f \; n) + \underbrace{S_{length}(n)}_{\in \; O(1)} + k
\end{equation*} 

Por lo tanto $ S_{mapS}(f \; n) \in O\left(\displaystyle\max_{i=0}^{n - 1}S_f(i)\right) $

%==============================================================================

\subsection{Función appendS}
Sea $n$ y $m$ la longitudes de los arreglos que reciben \texttt{appendS} como
argumento. 
\subsubsection{Trabajo:}

Podemos expresar la recurrencia para el trabajo de \texttt{appendS} de 
la siguiente manera:

\begin{equation*}
    W_{appendS}(n + m) = W_{tabulate}(f \; (n \; +\; m)) + \underbrace{W_{length}(n)}_{\in \; O(1)}
    + \underbrace{W_{length}(m)}_{\in \; O(1)} + k
\end{equation*}

Podemos ver fácilmente que la función que recibe \texttt{tabulateS} como
argumento, es $O(1)$ ya que simplemente toma un índice y devuelve el 
elemento de la lista que corresponde. Por lo tanto:

\begin{align*}
    W_{appendS}(n + m) \in O\left(\displaystyle\sum_{i=0}^{n + m - 1}W_f(i)\right)
    \Rightarrow W_{appendS}(n + m) \in O(n + m)
\end{align*}

\subsubsection{Profundidad:}
Podemos expresar la recurrencia para la profundidad de \texttt{appendS} de 
la siguiente manera:

\begin{equation*}
    S_{appendS}(n + m) = S_{tabulate}(f \; (n \; +\; m)) + \underbrace{S_{length}(n)}_{\in \; O(1)}
    + \underbrace{S_{length}(m)}_{\in \; O(1)} + k
\end{equation*}

Igualmente al trabajo $f \in O(1)$, por lo tanto:
\begin{align*}
    S_{appendS}(n + m) \in O\left(\displaystyle\max_{i=0}^{n + m - 1}S_f(i)\right)
    \Rightarrow S_{appendS}(n + m) \in O(1)
\end{align*}

%==============================================================================

\subsection{Función reduceS}

Sea $n$ la longitud de la lista que recibe \texttt{reduceS} como argumento, y $f$
una función de orden constante que toma la función como argumento.

\subsubsection{Trabajo:}

La recurrencia para el trabajo de \texttt{reduceS} la podemos expresar de
la siguiente manera:

\begin{equation*}
    W_{reduceS}(n) =  W_{reduceS}(\frac{n}{2}) + W_{contract}(n) + \underbrace{W_{length}(n)}_{\in \; O(1)} + k
\end{equation*}

Primero debemos saber que orden tiene la función \texttt{contract}, su recurrencia
tiene la forma:

\begin{equation*}
    W_{contract}(n) = W_{tabulateS}(f \; \frac{n}{2}) + \underbrace{W_{length}(n)}_{\in \; O(1)} + k
\end{equation*}

Por lo tanto, $W_{contract}(n) \in O\left(\displaystyle\sum_{i=0}^{\frac{n}{2}}W_f(i)\right) $
y como $f\in O(1)$ resulta $W_{contract}(n) \in O(n)$

Ahora utilizando el tercer caso del \textbf{Teorema Maestro}, podemos 
demostrar de forma análoga a la función equivalente en listas que
$W_{contract}(n) \in O(n)$.

\subsubsection{Profundidad:}

La recurrencia para el trabajo de \texttt{reduceS} podemos expresarla de la siguiente manera:

\begin{equation*}
    S_{reduceS}(n) =  S_{reduceS}(\frac{n}{2}) + S_{contract}(n) + \underbrace{W_{length}(n)}_{\in \; O(1)} + k
\end{equation*}

\begin{equation*}
    S_{contract}(n) = S_{tabulateS}(f \; \frac{n}{2}) + \underbrace{S_{length}(n)}_{\in \; O(1)} + k
\end{equation*}

$S_{contract}(n) \in O\left(\displaystyle\max_{i=0}^{\frac{n}{2}}S_f(i)\right) $
y como $f\in O(1)$ resulta $S_{contract}(n) \in O(1)$

Sabiendo esto.. bla bla bla
\begin{equation*}
    S_{reduceS}(n) =  S_{reduceS}(\frac{n}{2}) +  \underbrace{S_{contract}(n) }_{\in \; O(1)} +
    \underbrace{S_{length}(n)}_{\in \; O(1)} + k =  S_{reduceS}(\frac{n}{2}) + k'
\end{equation*}

Demostramos por inducción que $S_{reduceS}(n) \in O(\log{n})$

\begin{align*}
    S_{reduceS}(n) & = S_{reduceS}(\frac{n}{2}) + k' \\
                   & \leq c \cdot log{\frac{n}{2}} + k' \rightarrow HI\\
                   & = c \cdot \log{n} - c \cdot \underbrace{\log{2}}_{= 1} + k' \\
                   & = c \cdot \log{n} - c + k' \\
                   & \leq c \cdot \log{n} \iff c \geq k'
\end{align*}

Por lo tanto, $S_{reduceS}(n) \in O(\log{n})$.

%==============================================================================

\subsection{Función scanS}
\subsubsection{Trabajo:}

\subsubsection{Profundidad:}

\end{document}